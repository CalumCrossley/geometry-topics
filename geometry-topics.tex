\documentclass[a4paper]{article}

\usepackage[margin=1in]{geometry}

\usepackage{mathtools}
\usepackage{derivative}
\usepackage{enumitem} % enumeration label options
\usepackage{amsfonts} % \mathbb, ...
\usepackage{amssymb} % \subsetneq, \nmid, ...
\usepackage{stmaryrd} % \mapsfrom
\usepackage{yhmath} % \widehat
\usepackage{tikz-cd}

\DeclareMathOperator{\rad}{rad}

\newcommand{\m}{\mathfrak{m}}
\newcommand{\N}{\mathbb{N}}
\newcommand{\F}{\mathbb{F}}
\renewcommand{\P}{\mathbb{P}}
\newcommand{\A}{\mathbb{A}}
\newcommand{\Z}{\mathbb{Z}}
\newcommand{\Q}{\mathbb{Q}}
\newcommand{\R}{\mathbb{R}}
\newcommand{\C}{\mathbb{C}}

\title{Topics in Geometry (LSGNT) - Exercise solutions}
\author{Calum Crossley}
\date{2023-2024}

\begin{document}

\maketitle

\section{Spec and Proj}

\subsection{Exercises}

\begin{enumerate}

\item
\begin{enumerate}[label=(\alph*)]

\item 
A decomposition $V=V_1\cup V_2$ into subvarieties (Zariski
closed sets) $V_1$ and $V_2$ is the same thing as a pair of
disjoint Zariski open sets $V\setminus V_1$ and
$V\setminus V_2$. The decomposition is non-trivial iff the open
sets are non-empty.

\item
If $fg=0$ in $\C[V]$ then $V=V(f)\cup V(g)$, so irreducibility
implies one of $V(f)$ or $V(g)$ is $V$, meaning one of $f$ or
$g$ is zero. Hence $\C[V]$ is an integral domain if $V$ is
irreducible.

If $V=V_1\cup V_2$ and $V_i\ne V$ then some polynomial
$f_i$ vanishes on $V_i$ but not $V$, otherwise the equations
defining $V_i$ would include all of $V$. Then $f_1f_2$ vanishes
on $V_1\cup V_2=V$, so $\C[V]$ has zero-divisors.

\end{enumerate}

\item
\begin{enumerate}[label=(\alph*)]

\item 
Let $f$ be a non-constant polynomial with no roots over a non-algebraically
closed field, such as $x^2+1$ over $\R$. Then $V(f)=\emptyset$ so
$I_{V(f)}=(1)$, but $1\notin\rad(f)$ as $f$ is non-constant.

\item
Over $\F_p$ affine space $\A^n$ has finitely many points, so $I_{\A^n}$ consists
of the polynomials vanishing at these points, including for example
$\prod_{a_1\in\F_p}(x_1-a_1)=x_1^p-x_1$. We have
\begin{align*}
    I_{\A^n(\F_p)}
        &= \bigcap_{a_1,\ldots,a_n\in\F_p}(x_1-a_1,\ldots,x_n-a_n) \\
        &= \bigcap_{a_2,\ldots,a_n\in\F_p}(x_1^p-x_1,x_2-a_2,\ldots,x_n-a_n) \\
        &= (x_1^p-x_1,\ldots,x_n^p-x_n).
\end{align*}

\end{enumerate}

\item
Consider the affine variety $X=V(x_{n+1}f-1)\subseteq\A^{n+1}$. We have a
morphism $X\to\A^n\setminus V(f)$ by projection:
$(a_1,\ldots,a_{n+1})\mapsto(a_1,\ldots,a_n)$, since
$a_{n+1}f(a_1,\ldots,a_n)=1$ implies $f(a_1,\ldots,a_n)\ne0$. This has an
inverse given by $(a_1,\ldots,a_n)\mapsto(a_1,\ldots,a_n,1/f(a_1,\ldots,a_n))$,
which is also regular.

\item
Suppose $V\subseteq\A^n$ and $W\subseteq\A^m$, corresponding to quotient maps
$\C[x_1,\ldots,x_n]\to\C[V]$ and $\C[y_1,\ldots,y_m]\to\C[W]$. Choose a
lift of $\C[W]\to\C[V]$ and let $f_1,\ldots,f_m\in\C[x_1,\ldots,x_n]$
be the images of $y_1,\ldots,y_m$, as in the following diagram.
\begin{equation*}
\begin{tikzcd}[column sep=huge]
    &\C[W] \ar[r] &\C[V] \\
    &\C[y_1,\ldots,y_m] \ar[u,two heads] \ar[r,dashed,"f_1{,}\ldots{,}f_m"]
    &\C[x_1,\ldots,x_n] \ar[u,two heads]
\end{tikzcd}
\end{equation*}
Then $(x_1,\ldots,x_n)\mapsto(f_1,\ldots,f_m)$ is a morphism $\A^n\to\A^m$, and
since $I_W$ maps to $I_V$ in the diagram, it restricts to a morphism $V\to W$.
Moreover the pullback $\C[V]\to\C[W]$ given by composition with this map
precisely corresponds to substituting $f_1,\ldots,f_m$ for $y_1,\ldots,y_m$, so
from the diagram we recover the original homomorphism $\C[W]\to\C[V]$.

\item
If $\varphi:\C[V]\to\C[x]/(x^2)$ we have $\C[V]\to\C[x]/(x^2)\to\C[x]/(x)=\C$,
and the kernel of this map is a maximal ideal
$\m_p=\varphi^{-1}((x))\subseteq\C[V]$ corresponding to a point $p\in V$.
Moreover $\m_p/\m_p^2$ maps to $(x)/(x^2)=\C\cdot x$, giving an element of the
dual $(\m_p/\m_p^2)^\vee$ of $\m_p/\m_p^2$ considered as a
$\C[V]/\m_p=\C$-vector space.

\textbf{Claim 1:}
The data of a maximal ideal $\m_p\subseteq\C[V]$ and a functional
$v\in(\m_p/\m_p^2)^\vee$ determines a unique homomorphism $\C[V]\to\C[x]/(x^2)$
respecting the above construction.

\textbf{Proof:}
Define $\C[V]\to\C[x]/(x^2)$ by $f\mapsto f(p)+v(f-f(p))x$. This is clearly
$\C$-linear, and respects products since
\begin{equation*}
    fg-f(p)g(p) \equiv f(p)(g-g(p)) + g(p)(f-f(p)) \mod \m_p^2;
\end{equation*}
the difference being $(f-f(p))(g-g(p))$. By construction the preimage of $(x)$
is $\m_p$, and the induced map $\m_p/\m_p^2\to(x)/(x^2)=\C\cdot x$ is indeed
given by $v$.

\textbf{Claim 2:}
The space $(\m_p/\m_p^2)^\vee$ is canonically identified with the tangent space
to $V$ at $p$.

\textbf{Proof:}
Inuitively, elements of $(\m_p/\m_p^2)^\vee$ are functionals insensitive to
second order vanishing, which are first order differential operators;
differentiation along tangent vectors. Suppose
$V=V(f_1,\ldots,f_r)\subseteq\A^n$. Then the tangent space to $V$ at $p$ should
be given by
\begin{equation*}
    T_pV = \left\{(v_1,\ldots,v_n)\in\C^n
        : v_1\pdv{f_i}{x_1}(p)+\cdots+v_n\pdv{f_i}{x_n}(p)=0
        \text{ for $i=1,\ldots,r$}\right\}.
\end{equation*}
We define $\psi:(\m_p/\m_p^2)^\vee\to T_pV$ by
$v\mapsto(v(x_1-p_1),\ldots,v(x_n-p_n))$; evaluating the operator on
coordinates. This is well-defined since
\begin{equation*}
    (x_1-p_1)\pdv{f_i}{x_1}(p) + \cdots + (x_n-p_n)\pdv{f_i}{x_n}(p)
        \equiv f_i \mod \m_p^2,
\end{equation*}
and $f_i=0$ in $\C[V]$. It has an inverse given by
\begin{equation*}
    (v_1,\ldots,v_n) \mapsto
        \left(f\mapsto v_1\pdv{f}{x_1}(p)+\cdots+v_n\pdv{f}{x_n}(p)\right),
\end{equation*}
which is well-defined since elements of $\m_p^2$ map to zero by the product
rule, and multiples of $f_1,\ldots,f_r$ map to zero by the product rule and the
fact that $(v_1,\ldots,v_n)\in T_pV$. That this is indeed an inverse follows
from the fact that
\begin{equation*}
    (x_1-p_1)\pdv{f}{x_1}(p) + \cdots + (x_n-p_n)\pdv{f}{x_n}(p)
        \equiv f \mod \m_p^2.
\end{equation*}

\item
The corresponding homomorphism of coordinate rings $\C[x,y]/(y^2-x^3)\to\C[t]$
given by $x\mapsto t^2$, $y\mapsto t^3$ is not an isomorphism, since
$\C[t^2,t^3]\subsetneq\C[t]$.

However we have a set-theoretic inverse:
\begin{equation*}
    (x,y)\mapsto\begin{dcases*}
        x/y & if $y\ne0$ \\
        0 & otherwise.
    \end{dcases*}
\end{equation*}
Adjoining a point at infinity to both curves we get compact Hausdorff spaces in
the complex topology, so this inverse is even continuous with respect to the
complex topology.

\item
For $U_0\cap U_1$ we have $[1,x_1,x_2]=[1/x_1,1,x_2/x_1]$, giving
$(x_1,x_2)\mapsto(1/x_1,x_2/x_1)$.

For $U_0\cap U_2$ we have $[1,x_1,x_2]=[1/x_2,x_1/x_2,1]$, giving
$(x_1,x_2)\mapsto(1/x_2,x_1/x_2)$.

For $U_1\cap U_2$ we have $[x_0,1,x_2]=[x_0/x_2,1/x_2,1]$, giving
$(x_0,x_2)\mapsto(x_0/x_2,1/x_2)$.

\item
Suppose $f\in\C[x_1,\ldots,x_n]$ defines a hypersurface $V=V(f)\subseteq\A^n$.
We obtain a homogeneous polynomial $\hat f\in\C[x_1,\ldots,x_{n+1}]$ by
\begin{equation*}
    \hat f = x_{n+1}^{\deg f}\cdot f(x_1/x_{n+1},\ldots,x_n/x_{n+1}),
\end{equation*}
giving a projective hypersurface in $\P^n$ whose intersection with the affine
part $x_{n+1}\ne0$ is $V(f)$. It is in fact the closure of this affine part in
$\P^n$, since substituting $x_{n+1}=1$ and applying this process to a
homogeneous polynomial recovers the original polynomial up to a multiple of
$x_{n+1}$. When the codimension is higher homogenizing the generators of the
defining ideal like this is insufficient to cut out the projective closure; one
must homogenize all elements of the vanishing ideal. For example homogenizing
the generators of $V(x_1-x_2^2,x_1-x_3^3)$ gives
$V(x_1x_4-x_2^2,x_1x_4^2-x_3^3)$, whose points at infinity are given by

\item
\begin{enumerate}[label=(\alph*)]

\item 
Define a map $\P^1\to C$ by $[s:t]\mapsto[s^2:t^2:st]$. Now $xy=z^2$ implies
$[y:z]=[z:x]$ when both are defined, so $[x:y:z]\mapsto[y:z]$ and
$[x:y:z]\mapsto[z:x]$ glue to give an inverse $C\to\P^1$.

\item
Matrices of rank 1 are non-zero, and the rank is preserved by non-zero scalar
multiplication. Non-zero traceless $2\times2$ matrices are parametrized up to
scale by $\P^2$ as follows:
\begin{equation*}
    [x:y:z] \mapsto \begin{pmatrix}
        z & -x \\ y & -z
    \end{pmatrix},
\end{equation*}
and the rank is 1 iff the determinant $xy-z^2$ vanishes. Hence $C$ parametrizes
rank 1 traceless $2\times2$ matrices up to scale. From (a) we also get a
parametrization by $\P^1$:
\begin{equation*}
    [s:t] \mapsto \begin{pmatrix}
        st & -s^2 \\ t^2 & -st
    \end{pmatrix}.
\end{equation*}

\item
Over $\C$ the quadratic forms are determined up to equivalence by rank, so
replacing $xy-z^2$ by a non-degenerate quadratic form results in a curve
isomorphic to $C\cong\P^1$. A quadratic form of rank 2 results in a curve
isomorphic to $V(xy)$; two lines, and a quadratic form of rank 1 results in a
curve isomorphic to $V(x^2)$; one line (or a scheme-theoretic double line).

Different quadratic forms correspond to different special matrix forms as in
(b), for example
\begin{equation*}
    x^2 + y^2 = \det\begin{pmatrix}
        x & -y \\
        y & x
    \end{pmatrix}
\end{equation*}
and
\begin{equation*}
    xw - yz = \det\begin{pmatrix}
        x & y \\ z & w
    \end{pmatrix}.
\end{equation*}
The latter gives the Segre embedding $\P^1\times\P^1\to\P^3$,
$(X_1:X_2,Y_1:Y_2)\mapsto(X_1Y_1:X_1Y_2:X_2Y_1:X_2Y_2)$, whose image is
$V(XW-YZ)$.

\end{enumerate}

\item
\begin{enumerate}[label=(\alph*)]

\item
For $\epsilon\ne0$ we have $V_\epsilon\cong\A^1\setminus\{0\}$ via
$z\mapsto(z,\epsilon/z)$, which is
$(re^{i\theta},\epsilon r^{-1}e^{-i\theta})$
in polar coordinates. This forms a cylinder joining up the two real arcs
$\theta=0$ and $\theta=\pi$, with $(x,y)$ connecting to $(-x,-y)$ on the circle
$r=|x|$.

% TODO: picture

If we follow the circles $r=c$ for constant $c$ as $\epsilon\to0$ we get the
circles around the origin of the $x$-plane in $xy=0$. Similarly the circles
$r=c\epsilon$ converge to the circles around the origin of the $y$-plane.
Meanwhile the circle $r=\sqrt{|\epsilon|}$ contracts to the origin, so the
cylinder is pinched into two planes / discs connected at the origin.

\item

\end{enumerate}

\end{enumerate}

\section{Poincar\'e duality}

\subsection{Exercises}

\end{document}
